\chapter{Execution}
\label{sec_exec}

%%%%%%%%%%%%%%%%%%%%%%%%%%%%%%%%%%%%%%%%%%%%%%%%%%%%%%%%%%%%%%%%%%%%%%%%%%%%%%%%
\section{Environment}

LaBoGrid may be deployed in a variety of execution environments (desktop
computers, clusters, supercomputers, etc.). The only rule is: 1 worker per
computer, even if the computer has several cores (the worker will detect and
use them). The main reason for this is that the resources waste caused by the
execution of several JVMs on the same computer is avoided.

Therefore:
\begin{itemize}
	\item Executing LaBoGrid on a cluster implies that 1 JVM executing a worker
	is instantiated per computer.
	\item Executing LaBoGrid on a multi- or single-core computer implies the
	execution of 1 JVM executing a worker on this computer.
\end{itemize}


%%%%%%%%%%%%%%%%%%%%%%%%%%%%%%%%%%%%%%%%%%%%%%%%%%%%%%%%%%%%%%%%%%%%%%%%%%%%%%%%
\section{Launchers}

DiMaWo (see section~\ref{sec_arch_dimawo}) provides helper classes that can be
used to execute workers (i.e. classes providing a static ``main'' method).
These classes are called ``launchers''.

DiMaWo provides 2 worker launchers useful in the context of LaBoGrid: a
bootstrap worker launcher (first executed worker) and a normal worker launcher
(for all subsequent workers). These 2 launchers have respectively following
class names:

\begin{center}
\texttt{dimawo.exec.GenericBootstrapLauncher}\\
\texttt{dimawo.exec.GenericLauncher}
\end{center}

They execute a particular application by instantiating specific
classes implementing workers and master (see
section~\ref{sec_arch_dimawo_apps}) using a factory class provided by the user.
This factory may take parameters.


\subsection{Simulations}

LaBoGrid provides following factory class for the execution of flow simulations: 
%
\begin{center}
\texttt{laboGrid.LaBoGridFactory}\\
\end{center}
%
The factory takes 2 arguments:
%
\begin{enumerate}
	\item a configuration file (see chapter~\ref{sec_conf}),
	\item (optional) a power model file i.e. a file containing the result of
	previous benchmarks for a list of resources; the content of this file may be
	used by LaBoGrid when a power model is required for load balancing (see
	section~\ref{sec_conf_middle_load}).
\end{enumerate}

The bootstrap launcher requires at least 6 arguments:
%
\begin{enumerate}
	\item factory class name,
	\item port number the worker accepts connections on,
	\item working directory (the directory that will contain temporary and
	log files),
	\item maximum number of workers managed by a master-worker
	(master-worker included, see section~\ref{sec_arch_dimawo_model}),
	\item number of master-worker children per master-worker (see
	section~\ref{sec_arch_dimawo_model}),
	\item verbosity level (messages produced by agents with a verbosity level
	lower than given level are not printed, see
	section~\ref{sec_arch_dimawo_agents}),
\end{enumerate}
%
Following arguments are passed to given factory.

The normal launcher requires at least 8 arguments:
%
\begin{enumerate}
	\item factory class name,
	\item port number the worker accepts connections on,
	\item working directory (the directory that will contain temporary and
	log files),
	\item host name of a computer already executing a worker (generally the
	computer executing bootstrap worker),
	\item port number the already-running worker is accepting connections on,
	\item maximum number of workers managed by a master-worker
	(master-worker included, see section~\ref{sec_arch_dimawo_model}),
	\item number of master-worker children per master-worker (see
	section~\ref{sec_arch_dimawo_model}),
	\item verbosity level (messages with a verbosity level lower than given level
	are not printed),
\end{enumerate}
%
Following arguments are passed to given factory.


\subsection{Benchmarks}

Second optional argument of LaBoGrid's factory is a power model file. If it
describes well the computational power of the computers that execute LaBoGrid,
there is no need to execute benchmarks again. This is interesting when
simulations are mostly executed on the same set of computers or subsets of the
same set of computers: in this case, benchmarks are executed once for all and
their result re-used several times.

LaBoGrid provides a DiMaWo application which executes all the benchmarks
nee\-ded regarding a given configuration file on the computers the application
is executed by. Its factory class is:
%
\begin{center}
\texttt{laboGrid.ClusterBenchmarkFactory}
\end{center}
%
It takes 3 arguments:
\begin{enumerate}
	\item a LaBoGrid configuration file,
	\item the number of computers executing the application (i.e. the number
	of workers),
	\item an output file name (the name of the power model file to produce).
\end{enumerate}

Note that the content of a power model file is in a human readable format that
may be edited by hand.

